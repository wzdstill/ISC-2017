\documentclass{llncs}
\usepackage{url}
\usepackage{amssymb,amsmath,amsfonts,ntheorem}
\usepackage{enumerate}
\usepackage{enumitem}
\usepackage{chemarrow,extarrows}
\usepackage{graphicx}
\usepackage{enumerate}
\usepackage{enumitem}
\usepackage{framed}
\usepackage{xcolor,xspace}

\usepackage{float,subfigure}
\usepackage{etoolbox}
\usepackage{tabularx}
\usepackage[referable]{threeparttablex}
\newcommand{\N}{\ensuremath{\mathbb{N}}}
\newcommand{\Z}{\ensuremath{\mathbb{Z}}}
\newcommand{\R}{\ensuremath{\mathbb{R}}}
\newcommand{\Q}{\ensuremath{\mathbb{Q}}}
\newcommand{\poly}{\mathsf{poly}}
\newcommand{\prob}{\mathsf{Pr}}
\renewcommand\vec[1]{\ensuremath\boldsymbol{#1}}
\newcommand{\bool}{\{0,1\}}
\newcommand{\minusplus}{\{-1,1\}}
\newcommand{\xor}{\oplus}
\newcommand{\D}{\mathcal{D}}
\renewcommand{\S}{\mathcal{S}}
\newcommand{\mat}[1]{\mathbf{#1}}%
\newcommand{\T}{\mathbf{T}}



\newcommand{\SIS}{\ensuremath{\mathsf{SIS}}}
\newcommand{\LWE}{\ensuremath{\mathsf{LWE}}}
\newcommand{\RLWE}{\ensuremath{\mathsf{RLWE}}}
\newcommand{\Gapsvp}{\ensuremath{\mathsf{GapSVP}}}
\newcommand{\Sivp}{\ensuremath{\mathsf{SIVP}}}
\newcommand{\sampleleft}{\ensuremath{\mathsf{SampleLeft}}\xspace}
\newcommand{\sampleright}{\ensuremath{\mathsf{SampleRight}}\xspace}
\newcommand{\samplepre}{\ensuremath{\mathsf{SamplePre}}\xspace}
\newcommand{\extb}{\ensuremath{\mathsf{ExtBasis}}\xspace}
\newcommand{\randb}{\ensuremath{\mathsf{RandBasis}}\xspace}
\newcommand{\trapgen}{\ensuremath{\mathsf{TrapGen}}\xspace}
\newcommand{\G}{\ensuremath{\mathbf{G}}}

\newcommand{\Gr}{\ensuremath{\mathbb{G}}}
\newcommand{\F}{\ensuremath{\mathbb{F}}}

\newcommand{\ignore}[1]{}
\newcommand{\etal}{et al.\ }
\newcommand{\secparam}{\ensuremath{\lambda}\xspace}
\newcommand{\A}{\ensuremath{\mathcal{A}}\xspace}
\newcommand{\B}{\ensuremath{\mathcal{B}}\xspace}
\renewcommand{\O}{\ensuremath{\mathcal{O}}\xspace}
\newcommand{\Expt}{\ensuremath{\mathbf{Expt}}\xspace}
\newcommand{\ppt}{\ensuremath{\textsc{ppt}}\xspace}
\newcommand{\negl}{\ensuremath{\mathsf{negl}}\xspace}
\renewcommand{\H}{\ensuremath{\mathcal{H}}\xspace}
\newcommand{\X}{\ensuremath{\mathcal{X}}\xspace}
\newcommand{\Y}{\ensuremath{\mathcal{Y}}\xspace}
\newcommand{\hybrid}{\ensuremath{\mathsf{H}}\xspace}


\newcommand{\pk}{\ensuremath{\mathsf{pk}}\xspace}
\newcommand{\sk}{\ensuremath{\mathsf{sk}}\xspace}
\newcommand{\vk}{\ensuremath{\mathsf{vk}}\xspace}
\newcommand{\mpk}{\ensuremath{\mathsf{mpk}}\xspace}
\newcommand{\msk}{\ensuremath{\mathsf{msk}}\xspace}
\newcommand{\pp}{\ensuremath{\mathsf{pp}}\xspace}
\newcommand{\keygen}{\ensuremath{\mathsf{KeyGen}}\xspace}
\newcommand{\enc}{\ensuremath{\mathsf{Enc}}\xspace}
\newcommand{\dec}{\ensuremath{\mathsf{Dec}}\xspace}
\newcommand{\delegate}{\ensuremath{\mathsf{Delegate}}\xspace}
\newcommand{\setup}{\ensuremath{\mathsf{Setup}}\xspace}
\newcommand{\extract}{\ensuremath{\mathsf{Extract}}\xspace}
\newcommand{\fuzzy}{\ensuremath{\mathsf{Fuzzy}}\xspace}
\newcommand{\si}{\ensuremath{\mathsf{Sim}}\xspace}
\newcommand{\ct}{\ensuremath{\mathsf{ct}}\xspace}
\newcommand{\iO}{\ensuremath{i\mathcal{O}}\xspace}
\newcommand{\id}{\ensuremath{\mathsf{id}}\xspace}
\newcommand{\ID}{\ensuremath{\mathsf{ID}}\xspace}
\newcommand{\advantage}{\mathbf{Adv}}

\newcommand{\sign}{\ensuremath{\mathsf{Sign}}\xspace}
\newcommand{\siggen}{\ensuremath{\mathsf{SigGen}}\xspace}
\newcommand{\verify}{\ensuremath{\mathsf{Verify}}\xspace}


\newcommand{\IBE}{\ensuremath{\mathsf{IBE}}\xspace}
\newcommand{\ABE}{\ensuremath{\mathsf{ABE}}\xspace}
\newcommand{\IPE}{\ensuremath{\mathsf{IPE}}\xspace}
\newcommand{\FE}{\ensuremath{\mathsf{FE}}\xspace}
\newcommand{\PE}{\ensuremath{\mathsf{PE}}\xspace}

\patchcmd{\paragraph}{\itshape}{\bfseries\boldmath}{}{}


\newtheorem*{thm*}{Theorem}
\renewtheorem{theorem}{Theorem}[section]
\numberwithin{theorem}{section}
\renewtheorem{definition}[theorem]{Definition}%%
\renewtheorem{remark}[theorem]{Remark}
\renewtheorem{lemma}[theorem]{Lemma}
\renewtheorem{claim}[theorem]{Claim}
\newtheorem{cor}[theorem]{Corollary}
\newtheorem{asmp}[theorem]{Assumption}
\renewtheorem{proposition}[theorem]{Proposition}
\newtheorem{fact}[theorem]{Fact}
\pagestyle{plain}
\let\doendproof\endproof
\renewcommand\endproof{~\hfill$\qed$\doendproof}

\newcommand{\leo}[1]{{\color{cyan}[Leo: #1]}}
\newcommand{\todo}[1]{{\color{red}[TO-DO: #1]}}
\newcommand{\wang}[1]{{\color{red}[Wang: #1]}}
\newcommand{\sampled}{\mathsf{SampleD}}
\newcommand{\encode}{\mathsf{encode}}
\renewcommand{\T}{\mathsf{T}}
\begin{document}
\title{Compact Inner Product Encryption from LWE}%\\

\author{•}
\institute{}


\maketitle

\begin{abstract}
Predicate encryption provides fine-grained access control and has attractive applications. In this paper, We construct an compact inner product encryption scheme from the standard Learning with Errors (LWE) assumption that has compact public-key and achieves weakly attribute-hiding in the standard model. In particular, our scheme only needs two public matrices to support inner product over vector space $\Z_q^{\log \secparam}$, and $(\secparam / \log \secparam)$ public matrices to support vector space $\Z_q^{\secparam}$.

Our construction is the first compact functional encryption scheme based on lattice that goes beyond the very recent optimizations of public parameters in identity-based encryption setting. The main technique in our compact IPE scheme is a novel combination of IPE scheme of Agrawal, Freeman and Vaikuntanathan (Asiacrypt 2011), fully homomorphic encryption of Gentry, Sahai and Waters (Crypto 2013) and compact IBE scheme of Apon, Fan and Liu (Eprint 2016).
\end{abstract}



\section{Introduction}
Encryption has traditionally been regarded as a way to ensure confidentiality of an end-to-end communication. However, with the emergence of complex networks and cloud computing, recently the cryptographic community has been rethinking the notion of encryption to address security concerns that arise in these more complex environments. \emph{Functional encryption}~\cite{TCC:BonSahWat11,cryptoeprint:2010:556}, generalized from identity based encryption~\cite{C:Shamir84,C:BonFra01} and attribute based encryption~\cite{CCS:GPSW06,SP:BetSahWat07}, provides a satisfying solutions to this problem in theory. Two important features provided by functional encryption are fine grained access and computing on encrypted data. The fine grained access part is formalized as a cryptographic notion, named \emph{predicate encryption}~\cite{TCC:BonWat07,EC:KatSahWat08}. In predicate encryption system, each ciphertext $\ct$ is associated with an attribute $a$ and each secret key $\sk$ is associated with a predicate $f$. A user holding the key $\sk$ can decrypt ciphertext $\ct$ if and only if $f(a) = 0$. Moreover, the attribute $a$ is kept hidden.

With several significant improvements on quantum computing, the community is working intensively on developing applications whose security holds even against quantum attacks. Lattice-based cryptography, the most promising candidate against quantum attacks, has matured significantly since the early works of Ajtai~\cite{STOC:Ajtai96} and Regev~\cite{STOC:Regev05}. Most cryptographic primitives, ranging from basic public-key encryption (PKE)~\cite{STOC:Regev05} to more advanced schemes e.g., identity-based encryption (IBE)~\cite{EC:CHKP10,EC:AgrBonBoy10}, attribute-based encryption (ABE)~\cite{STOC:GorVaiWee13,EC:BGGHNS14}, predicate encryption~\cite{AC:AgrFreVai11,C:GorVaiWee15}, fully-homomorphic encryption (FHE)~\cite{STOC:Gentry09,FOCS:BraVai11,C:GenSahWat13}, etc., can be built from now canonical lattice hardness assumptions, such as Regev's Learning with Errors (LWE). From the above facts, we can draw the conclusion that our understanding about instantiating different cryptographic primitives based on lattices is quite well. However, for improving the efficiency of existent lattice-based construction, e.g. reducing the size of public parameters and ciphertexts, or simplifying the decryption algorithm, our understanding is limited. Besides the theoretical interests in shrinking the size of ciphertext, as the main motivation of studying functional encryption comes from its potential deployment in complex networks and cloud computing, thus the size of transmitted data is a bottleneck of current lattice-based constructions. Recently, there are several works~\cite{EC:Yamada16,EPRINT:ApoFanLiu16a,AC:KatYam16} about reducing the size of public parameters and ciphertexts in adaptively secure IBE setting. Combining all these, this brings us to the following open question:
\begin{center}
 \emph{Can we optimize the size of public parameters and ciphertexts of other functional encryption scheme beyond identity based encryption?}
\end{center}

\subsection{Our Contributions}
We positively answer the above question by proposing the first lattice-based compact inner product encryption (IPE). Rough speaking, in an IPE scheme, the secret key $\sk$ is associated with a predicate vector $\vec{v} \in \Z_q^t$ and the ciphertext is associated with an attribute vector $\vec{w} \in \Z_q^t$. The decryption works if and only if the inner product $\langle \vec{v}, \vec{w} \rangle = 0$. Despite this apparently restrictive structure, inner product predicates can support conjunction, subset and range queries on encrypted data~\cite{TCC:BonWat07}, as well as disjunctions, polynomial evaluation, and CNF and DNF formulas~\cite{EC:KatSahWat08}. Our construction can be summarized in the following informal theorem:
\begin{theorem}[Main]
Under the standard Learning with Errors assumption, there is an IPE scheme with weak attribute-hiding property supporting predicate/attribute vector of length $t = \log n$, where (1) the modulus $q$ is a prime of size polynomial in the security parameter $n$, (2) ciphertexts consist of a vector in $\Z_q^{2m + 1}$, $m$ is the lattice column dimension, and (3) the public parameters consists two matrices in $\Z_q^{n \times m}$ and a vector in $\Z_q^n$.
\end{theorem}

\begin{remark}\label{rem:sec}
\emph{
Our technique only allows us to prove a weak form of anonymity (``attribute hiding''). Specifically, given a ciphertext $\ct$ and a number of keys that do not decrypt $\ct$, the user cannot determine the attribute associated with $\ct$. In the strong form of attribute hiding, the user cannot determine the attribute associated with $\ct$ even when given keys that do decrypt $\ct$. The weakened form of attribute hiding we do achieve is nonetheless more that is required for $\ABE$ and should be sufficient for many applications of $\PE$. See Section $\textbf{2}$ for more detail.
}
\end{remark}

We can also extend our compact IPE construction to support $t = \poly(n)$ vectors. Let $t' = t / \log n$, our IPE construction supporting $\poly(n)$-length vectors can be stated in the following corollary:
\begin{cor}\label{cor}
Under the standard Learning with Errors assumption, there is an IPE scheme with weak attribute-hiding property supporting predicate/attribute vector of length $t = \poly(n)$, where (1) the modulus $q$ is a prime of size polynomial in the security parameter $n$, (2) ciphertexts consist of a vector in $\Z_q^{t'm + 1}$, and (3) the public parameters consists $(t' + 1)$ matrices in $\Z_q^{n \times m}$ and a vector in $\Z_q^n$.
\end{cor}
In addition to reducing the size of public parameters and ciphertexts, our decryption algorithm is computed in an Single-Instruction-Multiple-Data (SIMD) manner. In prior works~\cite{AC:AgrFreVai11,PKC:Xagawa13}, the decryption computes the inner product between the predicate vector and ciphertext by (1) decomposing the predicate vector, (2) multiplying-then-adding the corresponding vector bit and ciphertext, entry-by-entry. Our efficient decryption algorithm achieves the inner product by just one vector-matrix multiplication.

\subsection{Our Techniques}
Our high-level approach to compact inner product encryption from LWE begins by revisiting the first lattice-based IPE construction~\cite{AC:AgrFreVai11} and the novel fully homomorphic encryption proposed recently by Gentry, Sahai and Waters~\cite{C:GenSahWat13}.

\paragraph{The Agrawal-Freeman-Vaikuntanathan IPE.}We first review the construction of IPE in~\cite{AC:AgrFreVai11}. Their construction relies on the algebraic structure of ABB-IBE~\cite{EC:AgrBonBoy10} to solve ``lattice matching'' problem. Lattice matching means the lattice structure computed in decryption algorithm matches the structure used in key generation, and since the secret key is a \emph{short} trapdoor of the desired lattice, thus the decryption succeeds. To encode a predicate vector $\vec{v} \in \Z_q^t$ according to~\cite{AC:AgrFreVai11}, the key generation first computes the $r$-ary decomposition of each entry of $\vec{v}$ as $v_i = \sum_{j = 0}^k v_{ij} r^j$, and constructs the $\vec{v}$-specific lattice as
$$[\mat{A} | \mat{A}_{\vec{v}}] = [\mat{A} | \sum_{i = 1}^t \sum_{j = 0}^k v_{ij} \mat{A}_{ij}]$$
by ``mixing'' a \emph{long} public matrices $(\mat{A}, \{\mat{A}_{ij}\})$. The secret key $\sk_{\vec{v}}$ is a short trapdoor of lattice $\Lambda_q^{\bot}(\mat{A} | \sum_{i = 1}^t \sum_{j = 0}^k v_{ij} \mat{A}_{ij})$. To encode an attribute vector $\vec{w} \in \Z_q^{t}$, for $i \in [t], j \in [k]$, construct the $\vec{w}$-specific vector as
$$\vec{c}_{ij} = \vec{s}^\T ( \mat{A}_{ij} + r^j w_i \mat{B}) + \mathsf{noise}$$
for a randomly chosen vector $\vec{s}$ and a public matrix $\mat{B}$. To reduce the noise growth in the inner produce computation, decryption only needs to multiply-then-add the $r$-ary representation of $v_{ij}$ to its corresponding $\vec{c}_{ij}$, as
$$\sum_{i = 1}^t \sum_{j = 0}^k v_{ij}\vec{r}_{ij} = \vec{s}^\T (\sum_{i = 1}^t \sum_{j = 0}^k v_{ij} \mat{A}_{ij} + \langle\vec{v}, \vec{w}\rangle \mat{B}) + \mathsf{noise}$$
when $\langle\vec{v}, \vec{w}\rangle = 0$, the $(\langle\vec{v}, \vec{w}\rangle \mat{B})$ part vanishes, thus the lattice computed after inner produce matches the $\mat{A}_{\vec{v}}$ part in the key generation. Then the secret key $\sk_{\vec{v}}$ can be used to decrypt the ciphertext. Thererfore, the number of matrices in public parameters or vectors in ciphertext is quasilinear in the dimension of vectors.

\paragraph{Using GSW-FHE to compute inner product.} Recent progress in fully homomorphic encryption~\cite{C:GenSahWat13} makes us re-think the process of computing inner product. We wonder whether we can use GSW-FHE~\cite{C:GenSahWat13} along with its simplification~\cite{C:AlpPei14} to simplify the computing procedure. Recall ciphertext in GSW-FHE can be view in the form $\ct_x = \mat{A} \mat{R} + x \G$, where $\G$ is the ``gadget matrix'' as first (explicitly) introduced in the work~\cite{EC:MicPei12}. The salient point is that there is an efficiently computable function $\G^{-1}$, so that (1) $\ct_x \cdot \G^{-1}(y \G) = \ct_{xy}$, and (2) each entry in matrix $\G^{-1}(y \G)$ is just 0 or 1, and thus has small norm. These two nice properties can shrink the size of public parameters (ciphertext) from quasilinear to linear. In particular, to encoding a predicate vector $\vec{v} \in \Z_q^t$, we construct the $\vec{v}$-specific lattice as
$$[\mat{A} | \mat{A}_{\vec{v}}] = [\mat{A} | \sum_{i = 1}^t  \mat{A}_{i} \G^{-1}(v_i \G)]$$
where the number of public matrices is $t + 1$. To encode an attribute vector $\vec{w} \in \Z_q^{t}$, for $i \in [t]$, construct the $\vec{w}$-specific vector as
$$\vec{c}_i = \vec{s}^\T ( \mat{A}_i +  w_i \G) + \mathsf{noise}$$
Then, we can compute the inner product as
$$\sum_{i = 1}^t \vec{c}_i \cdot \G^{-1}(v_i \G) = \vec{s}^\T (\sum_{i = 1}^t  \mat{A}_{i} \G^{-1}(v_i \G) + \langle \vec{v}, \vec{w} \rangle \G) +  \mathsf{noise}$$
Since $\G^{-1}(v_i \G)$ is small norm, the decryption succeeds when $\langle \vec{v}, \vec{w} \rangle = 0$.

\paragraph{Achieving public parameters of two matrices.} Our final step is to bring the size of public parameters (or ciphertext) to constant for $(t = \log \secparam)$-length vectors. Inspired by recent work~\cite{EPRINT:ApoFanLiu16a} in optimizing size of public parameters in the IBE setting, we use their vector encoding method to further optimize our IPE construction. The vector encoding for encoding $\vec{v} \in \Z_q^t$ is
$$\mat{E}_{\vec{v}} = \big[v_1 \mat{I}_n | \cdots | v_t \mat{I}_n \big] \cdot
\G_{tn, \ell, m}$$
where $\G_{tn, \ell, m} \in \Z_q^{tn \times m}$ is the extended gadget matrix introduced in~\cite{EC:MicPei12,EPRINT:ApoFanLiu16a}. Then the $\vec{v}$-specific lattice becomes
$$\mat{A}_{\vec{v}} = \mat{A}_1 \cdot \G_{dn, \ell, m}^{-1}
 \Bigg(\begin{bmatrix}
v_{1}\mat{I}_n \\
\vdots \\
v_{d}\mat{I}_n
\end{bmatrix} \cdot \G_{n, 2, m}\Bigg) $$
and the $\vec{w}$-specific ciphertext becomes
$$\vec{c} = \vec{s}^\T ( \mat{A}_1 +  \mat{E}_{\vec{w}}) + \mathsf{noise}$$
The inner product can be computed in an SIMD way, as
$$ \vec{c} \cdot \G_{dn, \ell, m}^{-1}
 \Bigg(\begin{bmatrix}
v_{1}\mat{I}_n \\
\vdots \\
v_{d}\mat{I}_n
\end{bmatrix} \cdot \G_{n, 2, m}\Bigg) \approx \vec{s}^\T (\mat{A}_1\cdot \G_{dn, \ell, m}^{-1}
 \Bigg(\begin{bmatrix}
v_{1}\mat{I}_n \\
\vdots \\
v_{d}\mat{I}_n
\end{bmatrix} \G_{n, 2, m}\Bigg) +  \langle \vec{v}, \vec{w} \rangle \G_{n, 2, m})$$
As such, our final IPE system contains only two matrices $(\mat{A}, \mat{A}_1)$ (and a vector $\vec{u}$), and the ciphertext consists of two vectors. By carefully twisting the vector encoding and proof techniques shown in~\cite{AC:AgrFreVai11}, we show our IPE construction satisfies weakly attribute-hiding. Our IPE system can also be extended in a ``parallel repetition'' manner to support $(t = \secparam)$-length vectors, as Corollary~\ref{cor} states.


\subsection{Related Work}
In this section, we provide a comparison with the first IPE construction~\cite{AC:AgrFreVai11} and its follow-up improvement~\cite{PKC:Xagawa13}. In~\cite{PKC:Xagawa13}, Xagawa used the ``Full-Rank Difference encoding'', proposed in~\cite{EC:AgrBonBoy10} to map the vector $\Z_q^t$ to a matrix in $\Z_q^{n \times n}$. The size of public parameters (or ciphertext) in his scheme depends linearly on the length of predicate/attribute vectors, and the ``Full-Rank Difference encoding'' incurs more computation overhead than embedding GSW-FHE structure in IPE construction as described above. The detailed comparison is provided in Table~\ref{tab} for length parameter $t = \log \secparam$.

\newcolumntype{C}{>{\centering\arraybackslash}X}
\begin{table}[H]\label{tab}
\setlength\extrarowheight{2pt}
\setlength\tabcolsep{3pt}
\caption{Comparison of Lattice-based IPE Scheme} \label{tab:comparison}
 \begin{threeparttable}
\begin{tabularx}{\textwidth}{@{} l *{5}{C} @{}}
\hline
 Schemes
& \# of $\Z_q^{n \times m}$ mat. in $|\pp|$
&  \# of $\Z_q^m$ vec. in $|\ct|$
& LWE param $1 / \alpha$ \\ \hline
\cite{AC:AgrFreVai11} & $O(\secparam \log \secparam)$ & $O(\secparam \log \secparam)$ & $O(\secparam^{3.5})$ \\ \hline
\cite{PKC:Xagawa13} & $O(\secparam)$ & $O(\secparam)$ & $O(\secparam^4)$ \\ \hline
Ours  & 2 & 2 & $O(\secparam^{4} \log \secparam)$ \\ \hline
\end{tabularx}

 \end{threeparttable}
\end{table}







\section{Preliminaries}
\textbf{Notation.} Let $\secparam$ be the security parameter, and let $\ppt$ denote probabilistic polynomial time. We use bold uppercase letters to denote matrices ${\bf M}$, and bold lowercase letters to denote vectors $\vec{v}$. We write $\widetilde{\mat M}$ to denote the Gram-Schmidt orthogonalization of $\mat M.$ We write $[n]$ to denote the set $\{1,...,n\}$, and $|\vec{t}|$ to denote the number of bits in the string $\vec{t}$. We denote the $i$-th bit $\vec{s}$ by $\vec{s}[i]$. We say a function $\negl(\cdot): \N \rightarrow (0,1)$ is negligible, if for every constant $c \in \N$, $\negl(n) < n^{-c}$ for sufficiently large $n$.

\subsection{Inner Product Encryption}
We recall the syntax and security definition of \emph{inner product encryption} (IPE) ~\cite{EC:KatSahWat08,AC:AgrFreVai11}. IPE can be regarded as a generalization of predicate encryption. An IPE scheme $\Pi = (\setup, \keygen, \enc, \dec)$ can be described as follows:
\begin{description}
 \item $\setup(1^\secparam)$: On input the security parameter $\secparam$, the setup algorithm outputs public parameters $\pp$ and master secret key $\msk$.
 \item $\keygen(\msk, \vec{v})$: On input the master secret key $\msk$ and a predicate vector $\vec{v}$, the key generation algorithm outputs a secret key $\sk_{\vec{v}}$ for vector $\vec{v}$.
 \item $\enc(\pp, \vec{w}, \mu)$: On input the public parameter $\pp$ and an attribute/message pair $(\vec{w}, \mu)$, it outputs a ciphertext $\ct_{\vec{w}}$.
 \item $\dec(\sk_{\vec{v}}, \ct_{\vec{w}})$: On input the secret key $\sk_{\vec{v}}$ and a ciphertext $\ct_{\vec{w}}$, it outputs the corresponding plaintext $\mu$ if $\langle \vec{v}, \vec{w} \rangle = 0$; otherwise, it outputs $\bot$.
\end{description}

\begin{definition}[Correctness]\label{defn:cor}
We say the IPE scheme described above is correct, if for any $(\msk, \pp) \leftarrow \setup(1^\secparam)$, any message $\mu$, any predicate vector $\vec{v} \in \Z_q^d$, and attribute vector $\vec{w} \in \Z_q^d$ such that $\langle \vec{v}, \vec{w}\rangle = 0$, we have $\dec(\sk_{\vec{v}}, \ct_{\vec{w}}) = \mu$, where $\sk_{\vec{w}} \leftarrow \keygen(\msk, \vec{v})$ and $\ct_{\vec{v}} \leftarrow \enc(\pp, \vec{w}, \mu)$.
\end{definition}

\paragraph{Security.} For the weakly attribute-hiding property of IPE, we use the following experiment to describe it. Formally, for any $\ppt$ adversary $\A$, we consider the experiment $\Expt_{\A}^{\mathsf{IPE}}(1^\secparam)$:
\begin{itemize}[leftmargin=*]
 \item \textbf{Setup}: Adversary $\A$ sends two challenge attribute vectors $\vec{w}_{0}, \vec{w}_{1} \in \Z_q^d$ to challenger.  A challenger runs the $\setup(1^\secparam)$ algorithm, and sends back the master public key $\pp$.
 \item \textbf{Query Phase I}: Proceeding adaptively, the adversary $\A$ queries a sequence of predicate vectors $(\vec{v}_1,..., \vec{v}_m)$ subject to the restriction that $\langle \vec{v}_i, \vec{w}_{0} \rangle \neq 0$ and $\langle \vec{v}_i, \vec{w}_{1} \rangle \neq 0$. On the $i$-th query, the challenger runs $\sk_{\vec{v}_i} \rightarrow \keygen(\msk, \vec{v}_i),$ and sends the result $\sk_{\vec{v}_i}$ to $\A$.
 \item \textbf{Challenge}: Once adversary $\A$ decides that Query Phase I is over, he outputs two length-equal messages $(\mu^*_0, \mu^*_1)$ and sends them to challenger.  In response, the challenger selects a random bit $b^* \in \bool$, and sends the ciphertext $\ct^* \leftarrow \enc(\pp, \vec{w}_{b^*}, \mu_{b^*})$ to adversary $\A$.
 \item \textbf{Query Phase II}: Adversary $\A$ continues to issue secret key queries $(\vec{v}_{m + 1},..., \vec{v}_{n})$ adaptively, subject to the restriction that $\langle \vec{v}_i, \vec{w}_{0} \rangle \neq 0$ and $\langle \vec{v}_i, \vec{w}_{1} \rangle \neq 0$. The challenger responds by sending back keys $\sk_{\vec{v}_i}$ as in Query Phase I.
 \item \textbf{Guess}: Adversary $\A$ outputs a guess $b' \in \bool$.
\end{itemize}
We define the advantage of adversary $\A$ in attacking an IPE scheme $\Pi$ as:
$$\advantage_{\A}(1^\secparam) = \left|\Pr[b^* = b'] - \frac{1}{2}\right|,$$
\noindent where the probability is over the randomness of the challenger and adversary.

\begin{definition}[Weakly attribute-hiding]\label{defn:sec}
We say an IPE scheme $\Pi$ is weakly attribute-hiding against chosen-plaintext attacks in selective attribute setting, if for all $\ppt$ adversaries $\A$ engaging in experiment $\Expt_{\A}^{\mathsf{IPE}}(1^\secparam)$, we have
$$\advantage_{\A}(1^\secparam) \leq \negl(\secparam).$$
\end{definition}

\begin{remark}\label{rem:sec}
\emph{
We note that the experiment described in weakly attribute hiding definition is ``selective'', in the sense that the adversary must commit to its challenge attributes before seeing any secret keys. This selective security notion can be extended to adaptive one naturally by allowing adversary to output its challenge attributes in Challenge phase (or semi-adaptive one by allowing adversary to output its challenge attributes after seeing the public parameter, but before making any secret key queries). There are two more security notions, called {\em payload-hiding} and {\em attribute-hiding}, where payload-hiding requires $\vec{w}_0 = \vec{w}_1$, so any scheme that is weakly attribute hiding is payload hiding. Attribute-hiding requires that in secret key queries, predicate vectors $(\vec{v}_1,..., \vec{v}_n)$ satisfy $\langle \vec{v}_i, \vec{w}_0 \rangle = \langle \vec{v}_i, \vec{w}_1 \rangle$. As discussed in~\cite{cryptoeprint:2016:654}, the IPE construction in~\cite{AC:AgrFreVai11} dose not satisfy the attribute-hiding security. Our constructions described below suffer similar attacks. Therefore, it remains an interesting open problem to achieving attribute-hiding functional encryption from LWE, even for simple functionality such as inner product.
}
\end{remark}


\subsection{Lattice Background}
A full-rank $m$-dimensional integer lattice $\Lambda\subset\Z^m$ is a discrete additive subgroup whose linear span is $\R^m$. The basis of $\Lambda$ is a linearly independent set of vectors whose integer linear combinations are exactly $\Lambda$. Every integer lattice is generated as the $\Z$-linear combination of linearly independent vectors $\mat{B}=\{\vec{b}_1,...,\vec{b}_m\}\subset\Z^m$. For a matrix $\mat{A}\in\Z^{n\times m}_q$, we define the ``$q$-ary'' integer lattices:
$$\Lambda_q^{\bot}=\{\vec{e} \in \Z^m| \mat{A} \vec{e} = \vec{0} \bmod q\},\qquad  \Lambda_q^{\mat{u}}=\{\vec{e}\in\Z^m|\mat{A}\vec{e} = \vec{u} \bmod q\}$$
It is obvious that $\Lambda_q^{\vec{u}}$ is a coset of $\Lambda_q^{\bot}$.

Let $\Lambda$ be a discrete subset of $\Z^m$. For any vector $\vec{c}\in\R^m$, and any positive parameter $\sigma\in\R$, let $\rho_{\sigma, \vec{c}}(\vec{x})=\exp(-\pi||\vec{x}-\vec{c}||^2 / \sigma^2)$ be the Gaussian function on $\R^m$ with center $\vec{c}$ and parameter $\sigma$. Next, we let $\rho_{\sigma, \vec{c}}(\Lambda)=\sum_{\vec{x}\in\Lambda}\rho_{\sigma, \vec{c}}(\vec{x})$ be the discrete integral of $\rho_{\sigma, \vec{x}}$ over $\Lambda,$ and let $\D_{\Lambda, \sigma, \vec{c}}(\vec{y}):=\frac{\rho_{\sigma, \vec{c}}(\vec{y})}{\rho_{\sigma, \vec{c}}(\Lambda)}$. We abbreviate this as $\D_{\Lambda, \sigma}$ when $\vec{c}=\vec{0}.$

Let $S^m$ denote the set of vectors in $\R^{m}$ whose length is 1. Then the norm of a matrix $\mat{R} \in \R^{m \times m}$ is defined to be $\mathsf{sup}_{\vec{x} \in S^m} ||\mat{R} \vec{x}||$. Then we have the following lemma, which bounds the norm for some specified distributions.

\begin{lemma}[\cite{{EC:AgrBonBoy10}}]\label{lem:bound}
With respect to the norm defined above, we have the following bounds:
\begin{itemize}
 \item Let $\mat{R} \in \{-1, 1\}^{m \times m}$ be chosen at random, then we have $\prob[||\mat{R}|| > 12 \sqrt{2m}] < e^{-2m}$.
 \item Let $\mat{R}$ be sampled from $\D_{\Z^{m \times m}, \sigma}$, then we have $\prob [||\mat{R}|| > \sigma \sqrt{m}] < e^{-2m}$.
\end{itemize}
\end{lemma}


\paragraph{Randomness Extraction.} We will use the following lemma to argue the indistinghishability of two different distributions, which is a generalization of the leftover hash lemma proposed by Dodis et al. \cite{EC:DodReySmi04}.

\begin{lemma}[\cite{EC:AgrBonBoy10}] \label{lem:lhl}
Suppose that $m > (n + 1) \log q + \omega(\log n)$. Let $\mat{R} \in \{-1, 1\}^{m \times k}$ be chosen uniformly at random for some polynomial $k = k(n)$. Let $\mat{A}, \mat{B}$ be matrix chosen randomly from $\Z^{n \times m}_q, \Z^{n \times k}_q$ respectively. Then, for all vectors $\vec{w} \in \Z^m$, the two following distributions are statistically close:
$$(\mat{A}, \mat{A} \mat{R}, \mat{R}^\T \vec{w}) \approx (\mat{A}, \mat{B}, \mat{R}^\T \vec{w})$$
\end{lemma}

\paragraph{Learning With Errors.} The LWE problem was introduced by Regev~\cite{STOC:Regev05}, the works of~\cite{STOC:Regev05,STOC:Peikert09,STOC:BLPRS13} show that the LWE assumption is as hard as (quantum)
solving GapSVP and SIVP under various parameter regimes.
\begin{definition}[LWE]\label{defn:lwe}
For an integer $q = q(n) \geq 2$, and an error distribution $\chi = \chi(n)$ over $\Z_q$, the \emph{Learning With Errors problem $\LWE_{n, m, q, \chi}$} is to distinguish between the following pairs of distributions (e.g. as given by a sampling oracle $\mathcal{O}\in\{\mathcal{O}_{\vec{s}}, \mathcal{O}_{\$}\}$):
$$\{\mat{A}, \vec{s}^\T \mat{A} + \vec{x}\} \  \text{and} \ \{\mat{A}, \vec{u}\}$$
where $\mat{A} \overset{\$}{\leftarrow}\Z^{n \times m}_q$, $\vec{s} \overset{\$}{\leftarrow} \Z^n_q$, $\vec{u} \overset{\$}{\leftarrow} \Z^m_q$, and $\vec{x} \overset{\$}{\leftarrow} \chi^m$.
\end{definition}


\paragraph{Two-Sided Trapdoors and Sampling Algorithms.}

We will use the following algorithms to sample short vectors from specified lattices.

\begin{lemma}[\cite{STOC:GenPeiVai08,Alwen2010}] \label{lem:trapgen}
Let $q, n, m$ be positive integers with $q\geq 2$ and sufficiently large $m = \Omega(n \log q)$. There exists a $\ppt$ algorithm $\trapgen(q, n, m)$ that with overwhelming probability outputs a pair $(\mat{A}\in\Z_q^{n\times m}, \mat{T}_\mat{A} \in\Z^{m\times m})$ such that $\mat{A}$ is statistically close to uniform in $\Z_q^{n\times m}$ and $\mat{T}_\mat{A}$ is a basis for $\Lambda_q^{\bot}(\mat{A})$ satisfying
$$||\mat{T}_\mat{A}||\leq O(n\log q)\quad\mbox{and}\quad||\widetilde{\mat{T}_{\mat{A}}}||\leq O(\sqrt{n\log q})$$
except with $\mathsf{negl}(n)$ probability.
\end{lemma}

\begin{lemma}[\cite{STOC:GenPeiVai08,EC:CHKP10,EC:AgrBonBoy10}] \label{lem:samp}
Let $q>2, m>n.$ There are two sampling algorithms as follows:
\begin{itemize}[leftmargin=*]
 \item There is a \ppt\ algorithm $\sampleleft(\mat{A}, \mat{B}, \mat{T}_{\mat{A}}, \vec{u}, s)$, taking as input: (1) a rank-$n$ matrix $\mat{A}\in\Z_q^{n\times m},$ and any matrix $\mat{B}\in\Z_q^{n\times m_1}$, (2) a ``short'' basis $\mat{T}_{\mat{A}}$ for lattice $\Lambda_q^{\bot}(\mat{A})$, a vector $\vec{u}\in\Z_q^n$, (3) a Gaussian parameter $s > ||\widetilde{\mat{T}_{\mat{A}}}||\cdot\omega(\sqrt{\log(m+m_1)})$. Then outputs a vector $\vec r\in\Z^{m+m_1}$ distributed statistically close to $\D_{\Lambda_q^{\vec{u}}(\mat{F}), s}$ where $\mat{F}:=[\mat{A}|\mat{B}]$.

 \item There is a \ppt\ algorithm $\sampleright(\mat{A}, \mat{B}, \mat{R}, \mat{T}_{\mat{B}}, \vec{u}, s)$, taking as input: (1) a matrix $\mat{A}\in\Z_q^{n \times m},$ and a rank-$n$ matrix $\mat{B}\in\Z_q^{n\times m}$, a matrix $\mat{R}\in\Z_q^{m \times m},$ where $s_{\mat R} := ||\mat{R}|| = \sup_{\vec{x} : ||\vec{x}||=1}||\mat{R}\vec{x}||$, (2) a ``short'' basis $\mat{T}_{\mat{B}}$ for lattice $\Lambda_q^{\bot}(\mat{B}),$ a vector $\vec{u}\in\Z_q^n$, (3) a Gaussian parameter $s > ||\widetilde{\mat{T}_{\mat{B}}}||\cdot{s_{\mat R}}\cdot\omega(\sqrt{\log{m}})$. Then outputs a vector $\vec{r}\in\Z^{2m}$ distributed statistically close to $\D_{\Lambda_q^{\vec{u}}(\mat{F}), s}$ where $\mat{F}:=(\mat{A}|\mat{A}\mat{R} + \mat{B}).$

\end{itemize}
\end{lemma}

\paragraph{Gadget Matrix.} We now recall the gadget matrix~\cite{EC:MicPei12,C:AlpPei14}, and the extended gadget matrix technique appeared in~\cite{EPRINT:ApoFanLiu16a}, that are important to our construction.
\begin{definition}
Let $m = n \cdot \lceil\log q \rceil$, and define the gadget matrix
$$\mat{G}_{n, 2, m} = \vec{g} \otimes \mat{I}_n \in \Z_q^{n \times m}$$
where vector $\vec{g} = (1, 2, 4,..., 2^{\lfloor \log q \rfloor}) \in \Z_q^{\lceil \log q \rceil}$, and $\otimes$ denotes tenser product. We will also refer to this gadget matrix as ``powers-of-two'' matrix. We define the inverse function $\mat{G}^{-1}_{n, 2, m}: \Z_q^{n \times m} \rightarrow \bool^{m \times m}$ which expands each entry $a \in \Z_q$ of the input matrix into a column of size $\lceil \log q \rceil$ consisting of the bits of binary representations. We have the property that for any matrix $\mat{A} \in Z_q^{n \times m}$, it holds that $\G_{n, 2, m} \cdot \G^{-1}_{n, 2, m}(\mat{A}) = \mat{A}$.
\end{definition}
As mentioned by~\cite{EC:MicPei12} and explicitly described in~\cite{EPRINT:ApoFanLiu16a}, the results for $\mat{G}_{n, 2, m}$ and its trapdoor can be extended to other integer powers or mixed-integer products. In this direction, we give
a generalized notation for gadget matrices as follows:

For any modulus $q\ge 2,$ for integer base $2 \le b \le q,$ let $\vec{g}_{b}^T := \left[1, b, b^2, ..., b^{k_b-1}\right]\in\Z_q^{1\times k_b}$ for $k_b = \lceil\log_b{q}\rceil.$ (Note that the typical base-2 $\vec g^T$ is $\vec g_2^T.$) For row dimension $n$ and $b$ as before, we let $\mat G_{n, b} = \mat g_{b}^T\otimes\mat I_{n}\in\Z_q^{n\times nk_b}.$ The public trapdoor basis $\mat{T}_{\mat{G}_{n, b}}$ is given analogously. Similar to the above padding argument, $\mat G_{n, b}\in\Z_q^{n\times nk_b}$ can be padded into a matrix $\mat G_{n, b, m}\in\Z_{q}^{n\times m}$ for $m\geq  nk_b$ without increasing the norm of $\widetilde{\mat T_{\mat G_{n, b, m}}}$ from that of $\widetilde{\mat T_{\mat G_{n, b}}}.$

Following~\cite{PKC:Xagawa13}~and~\cite{C:AlpPei14}, we now introduce a related function -- the Batch Change-of-Base function $\mat{G}_{n', b', m'}^{-1}(\cdot)$ -- as follows:

For any modulus $q\ge 2,$ and for any integer base $2 \le b' \le q,$ let integer $k_{b'} := \lceil\log_{b'}(q)\rceil.$
For any integers $n'\ge 2$ and $m'\ge n'k_{b'}$ the function $\mat{G}_{n', b', m'}^{-1}(\cdot)$ takes as input a matrix from $\Z_q^{n'\times m'},$ first computes a matrix in $\{0, 1, ..., b'-1\}^{n'\log_{b'}(q)\times m'}$ using the typical $\mat G^{-1}$ procedure (except with base-$b'$ output), then pads with rows of zeroes as needed to form a matrix in $\{0, 1, ..., b'-1\}^{m'\times m'}.$ For example, the typical base-2 $\mat G^{-1} = \mat G^{-1}_{n, 2, m}$ takes $\Z_q^{n\times m}$ to $\bool^{m\times m}$ as expected.





\section{Our Construction}
In this section, we describe our compact IPE construction. Before diving into the details, we first revisit a novel encoding method implicitly employed in adaptively secure IBE setting in~\cite{EPRINT:ApoFanLiu16a}. Consider the vector space $\Z_q^d$. For vector $\vec{v} = (v_1,..., v_d) \in \Z_q^d$, we define the following encoding algorithm which maps a $d$-dimensional vector to an $n \times m$ matrix.
\begin{equation} \label{equ:enc}
\encode(\vec{v}) = \mat{E}_{\vec{v}} = \big[v_1 \mat{I}_n | \cdots | v_d \mat{I}_n \big] \cdot
\G_{dn, \ell, m}
\end{equation}
Similarly, we also define the encoding for an integer $a \in \Z_q$ as:
$\encode(a) = \mat{E}_a = a \G_{n, 2, m}.$ The above encoding supports the vector space operations naturally, and our compact IPE construction relies on this property.\\
We put the lattice background in the appendices for the limitation of space, which contains the description of related algorithms such as $\trapgen$ and $\sampleleft $.

\subsection{IPE Construction Supporting $\log(\secparam)$-length Attributes} \label{sec:log}
We describe our IPE scheme that each secret key is associated with a predicate vector $\vec{v} \in \Z^{d}_{q}$ (for some fixed $d = \log \secparam$), and each ciphertext will be associated with an attribute vector $\vec{w} \in \Z^{d}_{q}$. Decryption succeeds if and only if $\langle \vec{v},\vec{w} \rangle=0 \bmod q$. We further extend our IPE construction supporting $d = \poly(\secparam)$-length vectors in Section~\ref{sec:poly}. The description of $\Pi = (\setup, \keygen, \enc, \dec)$ is as follows:
\begin{itemize}[leftmargin=*]
 \item $\setup(1^\secparam, 1^d)$: On input the security parameter $\secparam$ and length parameter $d$, the setup algorithm first sets the parameters $(q, n, m, s)$ as below. We assume the parameters $(q, n, m, s)$ are implicitly included in both $\pp$ and $\msk$. Then it generates a random matrix $\mat{A} \in \Z_q^{n \times m}$ along with its trapdoor $\mat{T}_{\mat{A}} \in \Z_q^{m \times m}$, using $(\mat{A}, \mat{T}_{\mat{A}}) \leftarrow \trapgen(q, n, m)$. Next sample a random matrix $\mat{B} \in \Z_q^{n \times m}$ and a random vector $\vec{u} \in \Z_q^n$.  Output the public parameter $pp$ and master secret key $\msk$ as
  $$\pp = (\mat{A}, \mat{B}, \vec{u}), \qquad \msk = (\pp, \mat{T}_{\mat{A}})$$

 \item $\keygen(\msk, \vec{v})$: On input the master secret key $\msk$ and predictor vector $\vec{v} = (v_1,..., v_d) \in \Z_q^d$, the key generation algorithm first sets matrix $\mat{B}_{\vec{v}}$ as
 $$\mat{B}_{\vec{v}} = \mat{B} \cdot \G_{dn, \ell, m}^{-1}
 \Bigg(\begin{bmatrix}
v_{1}\mat{I}_n \\
\vdots \\
v_{d}\mat{I}_n
\end{bmatrix} \cdot \G_{n, 2, m}\Bigg) $$
 Then sample a low-norm vector $\vec{r}_{\vec{v}} \in \Z^{2m}$ using algorithm $\sampleleft(\mat{A}, \mat{B}_{\vec{v}}, \vec{u}, s)$, such that $[\mat{A} | \mat{B}_{\vec{v}}] \cdot \vec{r}_{\vec{v}} = \vec{u} \bmod q$. Output secret key $\sk_{\vec{v}} = \vec{r}_{\vec{v}}$.

 \item $\enc(\pp, \vec{w}, \mu)$: On input the public parameter $\pp$, an attribute vector $\vec{w} = (w_1,..., w_d) \in \Z_q^d$ and a message $\mu \in \{0,1\}$, the encryption algorithm first chooses a random vector $\vec{s} \in \Z_q^n$ and a random matrix $\mat{R} \in \{-1, 1\}^{m \times m}$. Then encode the attribute vector $\vec{w}$ as in Equation~(\ref{equ:enc})
 $$\mat{E}_{\vec{w}} = \big[w_1 \mat{I}_n | \cdots | w_d \mat{I}_n \big] \cdot
\G_{dn, \ell, m}$$
Let the ciphertext $\ct_{\vec{w}} = (\vec{c}_0, \vec{c}_1, c_2) \in \Z_q^{2m + 1}$ be
$$(\vec{c}_0, \vec{c}_1) = \vec{s}^\T [\mat{A} | \mat{B} + \mat{E}_{\vec{w}}] + (\vec{e}^\T_0, \vec{e}_0^T \mat{R}), \quad c_2 = \vec{s}^\T \vec{u} + e_1 + \lceil q / 2 \rceil \mu$$
where errors $\vec{e}_0 \leftarrow \D_{\Z^m, s}, e_1 \leftarrow \D_{\Z, s}$.

\item $\dec(\sk_{\vec{v}}, \ct_{\vec{w}})$: On input the secret key $\sk_{\vec{v}} = \vec{r}_{\vec{v}}$ and ciphertext $\ct_{\vec{w}} = (\vec{c}_0, \vec{c}_1, c_2)$, if $\langle \vec{v}, \vec{w} \rangle \neq 0 \bmod q$, then output $\bot$. Otherwise, first compute
$$\vec{c}'_1 = \vec{c}_1 \cdot \G_{dn, \ell, m}^{-1}
 \Bigg(\begin{bmatrix}
v_{1}\mat{I}_n \\
\vdots \\
v_{d}\mat{I}_n
\end{bmatrix} \cdot \G_{n, 2, m}\Bigg)$$
then output $\mathsf{Round}(c_2 - \langle (\vec{c}_0, \vec{c}'_1), \vec{r}_{\vec{v}} \rangle)$.
\end{itemize}


\paragraph{Correctness.} We prove the correctness of IPE scheme as follows:
\begin{lemma}\label{lem:cor}
The IPE scheme $\Pi$ described above is correct (c.f. Definition~\ref{defn:cor}).
\end{lemma}
\begin{proof}
When the predicate vector $\vec{v}$ and attribute vector $\vec{w}$ satisfies $\langle \vec{v}, \vec{w} \rangle = 0 \bmod q$, it holds that
\begin{align*}
\vec{c}'_1 & = \vec{s}^\T (\mat{B} + \mat{E}_{\vec{w}}) \G_{dn, \ell, m}^{-1}
 \Bigg(\begin{bmatrix}
v_{1}\mat{I}_n \\
\vdots \\
v_{d}\mat{I}_n
\end{bmatrix} \cdot \G_{n, 2, m}\Bigg) + \vec{e}'_0 \\
& = \vec{s}^\T \mat{B}_{\vec{v}} + \vec{s}^\T \langle \vec{v}, \vec{w} \rangle \cdot \G_{n, 2, m} + \vec{e}'_0= \vec{s}^\T \mat{B}_{\vec{v}} + \vec{e}'_0
\end{align*}
Therefore, during decryption, we have
\begin{align*}
\mu' &= \mathsf{Round}(c_2 - \langle(\vec{c}_0, \vec{c}'_1), \vec{r}_{\vec{v}} \rangle) \\
     &= \mathsf{Round}\bigg(\lceil q / 2 \rceil \mu + \underbrace{e_1 - \langle (\vec{e}_0, \vec{e}'_0), \vec{r}_{\vec{v}} \rangle}_{\text{small}}\bigg) = \mu \in \bool
\end{align*}
The third equation follows if $(e_1 - \langle (\vec{e}_0, \vec{e}'_0), \vec{r}_{\vec{v}} \rangle)$ is indeed small, which holds w.h.p. by setting the parameters appropriately below.
\end{proof}

\paragraph{Parameter Selection.}
To support $d = \log(\secparam)$-length predicate/attribute vectors, we set the system parameters according to Table~\ref{tab1}, where $\epsilon > 0$ is an arbitrarily small constant.
\vspace{-1em}
  \begin{table}[H]
  \centering
  \begin{tabular}{| c | c | c |}
   \hline
   Parameters & Description & Setting \\ \hline
   $\secparam$ & security parameter &  \\ \hline
   $n$ & lattice row dimension & $ \secparam$  \\ \hline
   $m$ & lattice column dimension & $ n^{1+\epsilon}$ \\ \hline
   $q$ & modulus & $n^{3+\epsilon}m$ \\ \hline
   $s$ & sampling and error width & $n^{1 + \epsilon} $ \\ \hline
   $\ell$ & integer-base parameter & $ n $ \\ \hline
   \end{tabular}
  \vspace{1em}
  \caption{$\log(\secparam)$-length IPE Parameters Setting}\label{tab1}
  \end{table}
  \vspace{-1em}
\noindent
 These values are chosen in order to satisfy the following constraints:
\begin{itemize}
\item To ensure correctness, we require $|e_1 - \langle (\vec{e}_0, \vec{e}'_0), \vec{r}_{\vec{v}} \rangle | < q/4$; Let $\vec{r}_{\vec{v}} = (\vec{r}_1, \vec{r}_2)$, here we can bound the dominating term:
$$ |\vec{e}_{0}^{'\T} \vec r_{2}| \leq || \vec{e}_{0}^{'\T}||  \cdot  ||\vec r_{2} || \approx  s \sqrt{m} d \ell \log_\ell q   \cdot s \sqrt{m}= s^2 m n^{1 + \epsilon} < q/4$$
\item For \sampleleft, we know $||\widetilde{\mat T_{\mat A}}|| = O(\sqrt{n\log(q)}),$ thus this requires that the sampling width $s$ satisfies $s > \sqrt{n\log(q)}\cdot\omega(\sqrt{\log(m)})$.
For $\sampleright$, we need $s > ||\widetilde{\mat T_{\G_{n, 2, m}}}|| \cdot ||\mat{R}|| \omega(\sqrt{\log m}) = n^{1 + \epsilon} \omega(\sqrt{\log m})$.
To apply Regev's reduction, we need $s > \sqrt{n}\omega(\log(n))$ ($s$ here is an absolute value, not a ratio). Therefore, we need $s > n^{1 + \epsilon}$
\item To apply the Leftover Hash Lemma, we need $m\ge (n+1)\log(q) + \omega(\log(n)).$
\end{itemize}

\paragraph{Security Proof.} Due to the limitation of the space, we put the security proof of our scheme in the appendices.


\section{IPE Construction Supporting $\poly(\secparam)$-length Vectors} \label{sec:poly}
In this part, we extend our IPE construction to support $t = \poly(\secparam)$-length vectors, which means the predicate and attribute vector are chosen in vector space $\Z_q^t$. Intuitively speaking, our construction described below can be regarded as a $t' = \lceil t / d \rceil$ ``parallel repetition'' version of IPE construction for $d = \log(\secparam)$-length vectors. In particular, we encode every $\log(\secparam)$ part of the attribute vector $\vec{v}$, and then concatenate these encoding together as the encoding of $\vec{v}$. The scheme $\Pi$ is described as follows:
\begin{itemize}[leftmargin=*]
 \item $\setup(1^\secparam, 1^t)$: On input security parameter $\secparam$ and length parameter $t$, the setup algorithm first sets the parameters $(q, n, m, s)$. We assume the parameters $(q, n, m, s)$ are implicitly included in both $\pp$ and $\msk$. Then it generates a random matrix $\mat{A} \in \Z_q^{n \times m}$ along with its trapdoor $\mat{T}_{\mat{A}} \in \Z_q^{m \times m}$, using $(\mat{A}, \mat{T}_{\mat{A}}) \leftarrow \trapgen(q, n, m)$. Next for $i \in [t']$, sample random matrix $\mat{B}_i \leftarrow \Z_q^{n \times m}$, then choose a random vector $\vec{u} \in \Z_q^n$.  Output the public parameter $pp$ and master secret key $\msk$ as
  $$\pp = (\mat{A}, \{\mat{B}_i\}_{i = 1}^{t'}, \vec{u}), \qquad \msk = (\pp, \mat{T}_{\mat{A}})$$

 \item $\keygen(\msk, \vec{v})$: On input the master secret key $\msk$ and a predicate vector $\vec{v} = (v_1,..., v_t) \in \Z_q^t$, the key generation first divides the vector $\vec{v}$ into $d$-coordinate vectors $\vec{v}_i \in \Z_q^d$ as
 $$\vec{v}_i = (v_{id + 1},..., v_{(i + 1)d}), \forall i \in [t']$$
 where we pad the vector $\vec{v}_{t'}$ with 0s if $t \leq d t'$. Then for $i \in [t']$, set matrix $\mat{B}_{\vec{v}_i}$ as
 $$\mat{B}_{\vec{v}_i} = \mat{B}_i \cdot \G_{dn, \ell, m}^{-1}
 \Bigg(\begin{bmatrix}
v_{id + 1}\mat{I}_n \\
\vdots \\
v_{(i + 1)d}\mat{I}_n
\end{bmatrix} \cdot \G_{n, 2, m}\Bigg)$$
Then sample a low-norm vector $\vec{r}_{\vec{v}} \in \Z^{(t' + 1)m}$ as
$$\vec{r}_{\vec{v}} \leftarrow \sampleleft(\mat{A}, [\mat{B}_{\vec{v}_1} | \cdots | \mat{B}_{\vec{v}_{t'}}], \vec{u}, s)$$
such that $[\mat{A} | \mat{B}_{\vec{v}_1} | \cdots | \mat{B}_{\vec{v}_{t'}}] \cdot \vec{r}_{\vec{v}} = \vec{u} \bmod q$. Output secret key $\sk_{\vec{v}} = \vec{r}_{\vec{v}}$.

\item $\enc(\pp, \vec{w}, \mu)$: On input the public parameter $\pp$, an attribute vector $\vec{w} = (w_1,..., w_t) \in \Z_q^t$ and a message $\mu \in \{0,1\}$, the encryption algorithm first chooses a random vector $\vec{s} \in \Z_q^m$ and $t'$ random matrices $\mat{R}_i \in \{-1, 1\}^{m \times m}$, for $i \in [t']$. Then parse the vector $\vec{w}$ into $d$-coordinate vectors $\vec{w}_i \in \Z_q^d$ as
 $$\vec{w}_i = (w_{id + 1},..., w_{(i + 1)d}), \forall i \in [t']$$
Next for $i \in [t']$, encode the attribute vector $\vec{w}$ as in Equation~(\ref{equ:enc})
 $$\mat{E}_{\vec{w}_i} = \big[w_1 \mat{I}_n | \cdots | w_d \mat{I}_n \big] \cdot
\G_{dn, \ell, m}$$
Let the ciphertext $\ct_{\vec{w}} = (\vec{c}_0, \{\vec{c}_{1i}\}_{i = 1}^{t'}, c_2) \in \Z_q^m \times \Z_q^{t'm} \times \Z_q$ be
$$\vec{c}_0 = \vec{s}^T \mat{A} + \vec{e}^\T_0, \quad \vec{c}_{1i} = \vec{s}^\T (\mat{B}_i + \mat{E}_{\vec{w}_i}) + \vec{e}_0^T \mat{R}_i, \quad c_2 = \vec{s}^\T \vec{u} + e_1 + \lceil q / 2 \rceil \mu$$
where errors $\vec{e}_0 \leftarrow \D_{\Z^m, s}, e_1 \leftarrow \D_{\Z, s}$.

\item $\dec(\sk_{\vec{v}}, \ct_{\vec{w}})$: On input the secret key $\sk_{\vec{v}} = \vec{r}_{\vec{v}}$ and ciphertext $\ct_{\vec{w}} = (\vec{c}_0, \vec{c}_1, c_2)$, if $\langle \vec{v}, \vec{w} \rangle \neq 0 \bmod q$, then output $\bot$. Otherwise, for $i \in [t']$, compute
$$\vec{c}'_{1i} = \vec{c}_{1i} \cdot \G_{dn, \ell, m}^{-1}
 \Bigg(\begin{bmatrix}
v_{id + 1}\mat{I}_n \\
\vdots \\
v_{(i + 1)d}\mat{I}_n
\end{bmatrix} \cdot \G_{n, 2, m}\Bigg)$$
Then let $\vec{c}'_1 = (\vec{c}'_{1i},..., \vec{c}'_{1t'})$, and output $\mathsf{Round}(c_2 - \langle (\vec{c}_0, \vec{c}'_1), \vec{r}_{\vec{v}} \rangle)$.
\end{itemize}
The correctness proof follows similar arguments of the $\log(\secparam)$-length counterpart (c.f. Lemma~\ref{lem:cor}), by  concatenating the ciphertext carefully as shown in the scheme. The parameters can also be set similarly as the previous one, thus we omit the parameter setting here.

\paragraph{Security Proof.} The security proof is similar to the previous one. We also put it in the appendices.



\section{Conclusion and Open Problems}
In this paper, we propose the first lattice-based inner product encryption whose size of public parameters and ciphertext does not linearly depend on the length of predicate/attribute vectors, at the expense of only constant growth of the modulus $q$ in terms of the security parameter $\secparam$. This leaves open two possibilities: (1) further optimize the system to achieve smaller ratio $|\ct, \pp| / t$ ($t$ is the length parameter), and (2) optimize the size of public parameters and ciphertexts of functional encryption for more general functions (instead of inner product), like ABE for polynomial-depth functions. Another important open problem about security, as discussed in Remark~\ref{rem:sec}, is to achieve attribute-hiding functional encryption from LWE, for simple functionality such as inner product or more general functionalities.





\bibliographystyle{plain}
\bibliography{abbrev3,crypto_crossref,extra,park,BCHK,BBL,ARW}

\begin{appendix}
\section{Lattice Background}
A full-rank $m$-dimensional integer lattice $\Lambda\subset\Z^m$ is a discrete additive subgroup whose linear span is $\R^m$. The basis of $\Lambda$ is a linearly independent set of vectors whose integer linear combinations are exactly $\Lambda$. Every integer lattice is generated as the $\Z$-linear combination of linearly independent vectors $\mat{B}=\{\vec{b}_1,...,\vec{b}_m\}\subset\Z^m$. For a matrix $\mat{A}\in\Z^{n\times m}_q$, we define the ``$q$-ary'' integer lattices:
$$\Lambda_q^{\bot}=\{\vec{e} \in \Z^m| \mat{A} \vec{e} = \vec{0} \bmod q\},\qquad  \Lambda_q^{\mat{u}}=\{\vec{e}\in\Z^m|\mat{A}\vec{e} = \vec{u} \bmod q\}$$
It is obvious that $\Lambda_q^{\vec{u}}$ is a coset of $\Lambda_q^{\bot}$.

Let $\Lambda$ be a discrete subset of $\Z^m$. For any vector $\vec{c}\in\R^m$, and any positive parameter $\sigma\in\R$, let $\rho_{\sigma, \vec{c}}(\vec{x})=\exp(-\pi||\vec{x}-\vec{c}||^2 / \sigma^2)$ be the Gaussian function on $\R^m$ with center $\vec{c}$ and parameter $\sigma$. Next, we let $\rho_{\sigma, \vec{c}}(\Lambda)=\sum_{\vec{x}\in\Lambda}\rho_{\sigma, \vec{c}}(\vec{x})$ be the discrete integral of $\rho_{\sigma, \vec{x}}$ over $\Lambda,$ and let $\D_{\Lambda, \sigma, \vec{c}}(\vec{y}):=\frac{\rho_{\sigma, \vec{c}}(\vec{y})}{\rho_{\sigma, \vec{c}}(\Lambda)}$. We abbreviate this as $\D_{\Lambda, \sigma}$ when $\vec{c}=\vec{0}.$

Let $S^m$ denote the set of vectors in $\R^{m}$ whose length is 1. Then the norm of a matrix $\mat{R} \in \R^{m \times m}$ is defined to be $\mathsf{sup}_{\vec{x} \in S^m} ||\mat{R} \vec{x}||$. Then we have the following lemma, which bounds the norm for some specified distributions.

\begin{lemma}[\cite{{EC:AgrBonBoy10}}]\label{lem:bound}
With respect to the norm defined above, we have the following bounds:
\begin{itemize}
 \item Let $\mat{R} \in \{-1, 1\}^{m \times m}$ be chosen at random, then we have $\prob[||\mat{R}|| > 12 \sqrt{2m}] < e^{-2m}$.
 \item Let $\mat{R}$ be sampled from $\D_{\Z^{m \times m}, \sigma}$, then we have $\prob [||\mat{R}|| > \sigma \sqrt{m}] < e^{-2m}$.
\end{itemize}
\end{lemma}


\paragraph{Randomness Extraction.} We will use the following lemma to argue the indistinghishability of two different distributions, which is a generalization of the leftover hash lemma proposed by Dodis et al. \cite{EC:DodReySmi04}.

\begin{lemma}[\cite{EC:AgrBonBoy10}] \label{lem:lhl}
Suppose that $m > (n + 1) \log q + \omega(\log n)$. Let $\mat{R} \in \{-1, 1\}^{m \times k}$ be chosen uniformly at random for some polynomial $k = k(n)$. Let $\mat{A}, \mat{B}$ be matrix chosen randomly from $\Z^{n \times m}_q, \Z^{n \times k}_q$ respectively. Then, for all vectors $\vec{w} \in \Z^m$, the two following distributions are statistically close:
$$(\mat{A}, \mat{A} \mat{R}, \mat{R}^\T \vec{w}) \approx (\mat{A}, \mat{B}, \mat{R}^\T \vec{w})$$
\end{lemma}

\paragraph{Learning With Errors.} The LWE problem was introduced by Regev~\cite{STOC:Regev05}, the works of~\cite{STOC:Regev05,STOC:Peikert09,STOC:BLPRS13} show that the LWE assumption is as hard as (quantum)
solving GapSVP and SIVP under various parameter regimes.
\begin{definition}[LWE]\label{defn:lwe}
For an integer $q = q(n) \geq 2$, and an error distribution $\chi = \chi(n)$ over $\Z_q$, the \emph{Learning With Errors problem $\LWE_{n, m, q, \chi}$} is to distinguish between the following pairs of distributions (e.g. as given by a sampling oracle $\mathcal{O}\in\{\mathcal{O}_{\vec{s}}, \mathcal{O}_{\$}\}$):
$$\{\mat{A}, \vec{s}^\T \mat{A} + \vec{x}\} \  \text{and} \ \{\mat{A}, \vec{u}\}$$
where $\mat{A} \overset{\$}{\leftarrow}\Z^{n \times m}_q$, $\vec{s} \overset{\$}{\leftarrow} \Z^n_q$, $\vec{u} \overset{\$}{\leftarrow} \Z^m_q$, and $\vec{x} \overset{\$}{\leftarrow} \chi^m$.
\end{definition}


\paragraph{Two-Sided Trapdoors and Sampling Algorithms.}

We will use the following algorithms to sample short vectors from specified lattices.

\begin{lemma}[\cite{STOC:GenPeiVai08,Alwen2010}] \label{lem:trapgen}
Let $q, n, m$ be positive integers with $q\geq 2$ and sufficiently large $m = \Omega(n \log q)$. There exists a $\ppt$ algorithm $\trapgen(q, n, m)$ that with overwhelming probability outputs a pair $(\mat{A}\in\Z_q^{n\times m}, \mat{T}_\mat{A} \in\Z^{m\times m})$ such that $\mat{A}$ is statistically close to uniform in $\Z_q^{n\times m}$ and $\mat{T}_\mat{A}$ is a basis for $\Lambda_q^{\bot}(\mat{A})$ satisfying
$$||\mat{T}_\mat{A}||\leq O(n\log q)\quad\mbox{and}\quad||\widetilde{\mat{T}_{\mat{A}}}||\leq O(\sqrt{n\log q})$$
except with $\mathsf{negl}(n)$ probability.
\end{lemma}

\begin{lemma}[\cite{STOC:GenPeiVai08,EC:CHKP10,EC:AgrBonBoy10}] \label{lem:samp}
Let $q>2, m>n.$ There are two sampling algorithms as follows:
\begin{itemize}[leftmargin=*]
 \item There is a \ppt\ algorithm $\sampleleft(\mat{A}, \mat{B}, \mat{T}_{\mat{A}}, \vec{u}, s)$, taking as input: (1) a rank-$n$ matrix $\mat{A}\in\Z_q^{n\times m},$ and any matrix $\mat{B}\in\Z_q^{n\times m_1}$, (2) a ``short'' basis $\mat{T}_{\mat{A}}$ for lattice $\Lambda_q^{\bot}(\mat{A})$, a vector $\vec{u}\in\Z_q^n$, (3) a Gaussian parameter $s > ||\widetilde{\mat{T}_{\mat{A}}}||\cdot\omega(\sqrt{\log(m+m_1)})$. Then outputs a vector $\vec r\in\Z^{m+m_1}$ distributed statistically close to $\D_{\Lambda_q^{\vec{u}}(\mat{F}), s}$ where $\mat{F}:=[\mat{A}|\mat{B}]$.

 \item There is a \ppt\ algorithm $\sampleright(\mat{A}, \mat{B}, \mat{R}, \mat{T}_{\mat{B}}, \vec{u}, s)$, taking as input: (1) a matrix $\mat{A}\in\Z_q^{n \times m},$ and a rank-$n$ matrix $\mat{B}\in\Z_q^{n\times m}$, a matrix $\mat{R}\in\Z_q^{m \times m},$ where $s_{\mat R} := ||\mat{R}|| = \sup_{\vec{x} : ||\vec{x}||=1}||\mat{R}\vec{x}||$, (2) a ``short'' basis $\mat{T}_{\mat{B}}$ for lattice $\Lambda_q^{\bot}(\mat{B}),$ a vector $\vec{u}\in\Z_q^n$, (3) a Gaussian parameter $s > ||\widetilde{\mat{T}_{\mat{B}}}||\cdot{s_{\mat R}}\cdot\omega(\sqrt{\log{m}})$. Then outputs a vector $\vec{r}\in\Z^{2m}$ distributed statistically close to $\D_{\Lambda_q^{\vec{u}}(\mat{F}), s}$ where $\mat{F}:=(\mat{A}|\mat{A}\mat{R} + \mat{B}).$

\end{itemize}
\end{lemma}

\section{Security Proof}
\subsection{Security Proof of Our Main Scheme}
In this part, we show the weakly attribute-hiding property of our IPE construction. We adapt the simulation technique in~\cite{AC:AgrFreVai11} by plugin the encoding of vectors. Intuitively, to prove the theorem we define a sequences of hybrids against adversary $\mathcal{A}$ in the weak attribute-hiding experiment. The adversary $\A$ outputs two attribute vectors $\vec{w}_{0}$ and $\vec{w}_{1}$ at the beginning of each game, and at some point outputs two messages $\mu_{0},\mu_{1}$. The first and last games correspond to real security game with challenge ciphertexts $\enc(\pp,\vec{w}_{0},\mu_{0})$ and $\enc(\pp,\vec{w}_{1},\mu_{1})$ respectively. In the intermediate games we use the ``alternative'' simulation algorithms $(\si.\setup,\si.\keygen, \si.\enc)$. During the course of the game the adversary can only request keys for predicate vector $\vec{v}_i$ such that $\langle \vec{v}_i, \vec{w}_{0} \rangle \neq 0$ and $\langle \vec{v}_i, \vec{w}_{1} \rangle \neq 0$.

We first define the simulation algorithms $(\si.\setup,\si.\keygen, \si.\enc)$ in the following:
\begin{itemize}[leftmargin=*]
 \item $\si.\setup(1^\secparam, 1^d, \vec{w}^*)$: On input the security parameter $\lambda$, the length parameter $d$, and an attribute vector $\vec{w}^{*} \in \Z^{d}_{q}$, the simulation setup algorithm first chooses a random matrix $\mat{A} \leftarrow \Z_q^{n \times m}$ and a random vector $\vec{u} \leftarrow \Z_q^n$. Then set matrix
 $$\mat{B} = \mat{A} \mat{R}^* - \mat{E}_{\vec{w}^*}, \quad \mat{E}_{\vec{w}^*} = \big[w^*_1 \mat{I}_n | \cdots | w^*_d \mat{I}_n \big] \cdot
\G_{dn, \ell, m}$$
where matrix $\mat{R}^*$ is chosen randomly from $\{-1, 1\}^{m \times m}$. Output $\pp = (\mat{A}, \mat{B}, \vec{u})$ and $\msk = \mat{R}^*$.

 \item $\si.\keygen(\msk, \vec{v})$: On input the master secret key $\msk$ and a vector $\vec{v} \in \Z_q^d$, the simulation key generation algorithm sets matrix $\mat{R}_{\vec{v}}$ and  $\mat{B}_{\vec{v}}$ as
 $$\mat{R}_{\vec{v}} =
 \Bigg(\begin{bmatrix}
v_{1}\mat{I}_n \\
\vdots \\
v_{d}\mat{I}_n
\end{bmatrix} \cdot \G_{n, 2, m}\Bigg), \quad \mat{B}_{\vec{v}} = \mat{B} \cdot \G_{dn, \ell, m}^{-1}(\mat{R}_{\vec{v}})$$
By further unfolding $\mat{B}_{\vec{v}}$, we have
\begin{align*}
\mat{B}_{\vec{v}} & = \mat{A} \mat{R}^*\G_{dn, \ell, m}^{-1}(\mat{R}_{\vec{v}}) -  \mat{E}_{\vec{w}^*} \cdot \G_{dn, \ell, m}^{-1}(\mat{R}_{\vec{v}}) \\
 & = \mat{A} \mat{R}^*\G_{dn, \ell, m}^{-1}(\mat{R}_{\vec{v}}) - \langle \vec{v}, \vec{w}^* \rangle \cdot \G_{n, 2, m}
\end{align*}
Then sample a low-norm vector $\vec{r}_{\vec{v}} \in \Z^{2m}$ using algorithm
$$\vec{r}_{\vec{v}} \leftarrow \sampleright(\mat{A}, \langle \vec{v}, \vec{w}^* \rangle\G_{n, 2, m}, \mat{R}^*\G_{dn, \ell, m}^{-1}(\mat{R}_{\vec{v}}), \mat{T}_{\G_{n, 2, m}}  \vec{u}, s)$$
such that $[\mat{A} | \mat{B}_{\vec{v}}] \cdot \vec{r}_{\vec{v}} = \vec{u} \bmod q$. Output secret key $\sk_{\vec{v}} = \vec{r}_{\vec{v}}$.

 \item $\si.\enc(\pp, \vec{w}^*, \mu)$: The simulation encryption algorithm is the same as the counterpart in the scheme, except the matrix $\mat{R}^*$ is used in generating the ciphertext instead of sampling a random matrix $\mat{R} \in \{-1, 1\}^{m \times m}$.
\end{itemize}

\begin{theorem}\label{thm:sec}
Assuming the hardness of $(n, q, \chi)$-LWE assumption, the IPE scheme described above is weakly attribute-hiding (c.f. Definition~\ref{defn:sec}).
\end{theorem}
\begin{proof}
The sequence of hybrids are described as follows:
\begin{itemize}[leftmargin=*]
 \item\textbf{Hybrid} $\hybrid_0$: The challenger runs $\setup$, answers $\A$'s secret key queries using $\keygen$, and generates the challenge ciphertext $\ct^*$ using $\enc$ with attribute $\vec{w}_{0}$ and message $\mu_{0}$.
 \item\textbf{Hybrid} $\hybrid_1$: The challenger runs $\si.\setup$ with $\vec{w}^{*}=\vec{w}_{0}$, and answers $\A$'s secret key queries using $\si.\keygen$. The challenger generates the challenge ciphertext $\ct^*$ using $\si.\enc$ with attribute  $\vec{w}_{0}$ and message $\mu_{0}$.
 \item\textbf{Hybrid} $\hybrid_2$: The challenger runs $\si.\setup$ with $\vec{w}^{*}=\vec{w}_{0}$, and answers $\A$'s secret key queries using $\si.\keygen$. The challenger generates the challenge ciphertext $\ct^*$ by choosing a uniformly random element of the ciphertext space.
 \item\textbf{Hybrid} $\hybrid_3$: The challenger runs $\si.\setup$ with $\vec{w}^{*}=\vec{w}_{1}$, and answers $\A$'s secret key queries using $\si.\keygen$. The challenger generates the challenge ciphertext $\ct^*$ by choosing a uniformly random element of the ciphertext space.
 \item\textbf{Hybrid} $\hybrid_4$: The challenger runs $\si.\setup$ with $\vec{w}^{*}=\vec{w}_{1}$, and answers $\A$'s secret key queries using $\si.\keygen$. The challenger generates the challenge ciphertext $\ct^*$ using $\si.\enc$ with attribute  $\vec{w}_{1}$ and message $\mu_{1}$.
 \item\textbf{Hybrid} $\hybrid_5$: The challenger runs $\setup$, answers $\A$'s secret key queries using $\keygen$, and generates the challenge ciphertext $\ct^*$ using $\enc$ with attribute $\vec{w}_{1}$ and message $\mu_{1}$.
\end{itemize}
Regarding the above hybrids, we have the following lemmas.

\begin{lemma}\label{lem:hybrid0}
The view of adversary $\A$ in hybrid $\hybrid_0$ (or $\hybrid_4$) is statistically close to the view of adversary $\A$ in hybrid $\hybrid_1$ (or $\hybrid_5$).
\end{lemma}


\begin{lemma}\label{lem:hybrid1}
Assuming the hardness of $(n, q, \chi)$-LWE assumption, the view of adversary $\A$ in hybrid $\hybrid_1$ (or $\hybrid_3$) is computationally close to the view of adversary $\A$ in hybrid $\hybrid_2$ (or $\hybrid_4$).
\end{lemma}


\begin{lemma}\label{lem:hybrid2}
The view of adversary $\A$ in hybrid $\hybrid_2$ is statistically close to the view of adversary $\A$ in hybrid $\hybrid_3$.
\end{lemma}

The proofs of the above lemmas are analogous to the counterparts in~\cite{AC:AgrFreVai11}. Due to the space limit, we include these proofs in the full version.

\paragraph{Completing the proof.}Suppose that there is an $\ppt$ adversary $\A$ that wins the weakly attribute-hiding experiment. Let $\A^i$ be the output of $\A$ interacting with hybrid $\hybrid_i$, then we have
$$|\prob[\A^0 = 1] - \prob[\A^5 = 1]| \geq 1 / \poly(\secparam)$$
By a standard argument, we have for $i = 0,..., 4$
$$|\prob[\A^i = 1] - \prob[\A^{(i + 1)} = 1]| \geq 1 / \poly(\secparam)$$
Since $\A$ is polynomial time, the above inequalities contradict the statistical arguments in the proof of Lemma~\ref{lem:hybrid0},~\ref{lem:hybrid2}. By Lemma~\ref{lem:hybrid1}, adversary $\A$ can be used to solve LWE instance, which also contradicts the hardness of LWE assumption.

\end{proof}

\subsection{Security Proof of poly(��)-length Vectors Version}
Put simply, to prove the weakly attribute-hiding property (c.f. Definition~\ref{defn:sec}), we describe a sequence of hybrids analogous to those in the proof of Theorem~\ref{thm:sec}, and use similar statistical arguments or LWE assumption to argue the indistinguishability of two consecutive hybrids. However, the algorithm $\si.\keygen$ is different from the prior one, since the $\sampleright$ algorithm only supports sampling from a special form of lattices, and the lattice we use to encode the $\poly$-length predicate vector $\vec{v}$ obviously is not consistent with that. The $\si.\keygen$ algorithm can be described as follows:
\begin{description}
 \item $\si.\keygen(\msk, \vec{v})$: On input the master secret key $\msk$ and a predicate vector $\Z_q^t$, the simulation first divides the vector $\vec{v}$ into $d$-coordinate vectors $\vec{v}_i \in \Z_q^d$ as
 $$\vec{v}_i = (v_{id + 1},..., v_{(i + 1)d}), \forall i \in [t']$$
 Then for $i \in [t']$, set matrix $\mat{R}_{\vec{v}_i}$ and  $\mat{B}_{\vec{v}_i}$ as
 $$\mat{R}_{\vec{v}_i} =
 \Bigg(\begin{bmatrix}
v_{id + 1}\mat{I}_n \\
\vdots \\
v_{(i + 1)d}\mat{I}_n
\end{bmatrix} \cdot \G_{n, 2, m}\Bigg), \quad \mat{B}_{\vec{v}_i} = \mat{B}_i \cdot \G_{dn, \ell, m}^{-1}(\mat{R}_{\vec{v}_i})$$
By plugin $\mat{B}_i = \mat{A} \mat{R}^*_i - \mat{E}_{\vec{w}^*_i}$, where $\vec{w}^* = (\vec{w}_1^*,..., \vec{w}_{t'}^*)$, we have
\begin{align*}
\mat{B}_{\vec{v}_i} & = \mat{A} \mat{R}_i^*\G_{dn, \ell, m}^{-1}(\mat{R}_{\vec{v}_i}) -  \mat{E}_{\vec{w}_i^*} \cdot \G_{dn, \ell, m}^{-1}(\mat{R}_{\vec{v}_i}) \\
 & = \mat{A} \mat{R}^*\G_{dn, \ell, m}^{-1}(\mat{R}_{\vec{v}_i}) - \langle \vec{v}_i, \vec{w}_i^* \rangle \cdot \G_{n, 2, m}
\end{align*}
Since $\langle \vec{v}, \vec{w}^* \rangle = \sum_{i  = 1}^{t'} \langle \vec{v}_i, \vec{w}_i^* \rangle \neq 0 \bmod q$, then with overwhelming probability there must exists at least one index $i \in [t']$, such that $\langle \vec{v}_i, \vec{w}_i^* \rangle \neq 0 \bmod q$. Pick the smallest index $k \in [t']$ such that $\langle \vec{v}_k, \vec{w}_k^* \rangle \neq 0 \bmod q$. Next sample $(t' - 1)$ discrete Gaussian vectors $\vec{r}_i \in \D_{\Z^m, s}$, and sample $(\vec{r}_0, \vec{r}_k) \in \Z^{2m}$ using
$$\sampleright(\mat{A}, \langle \vec{v}_k, \vec{w}_k^* \rangle\G_{n, 2, m}, \mat{R}_k^*\G_{dn, \ell, m}^{-1}(\mat{R}_{\vec{v}_k}), \mat{T}_{\G_{n, 2, m}}  \vec{u} - \sum_{i \neq k} \mat{B}_{\vec{v}_i} \vec{r}_i , s)$$
Therefore, it holds $[\mat{A} | \mat{B}_{\vec{v}_1} | \cdots | \mat{B}_{\vec{v}_{t'}}] \cdot (\vec{r}_0,..., \vec{r}_{t'}) = \vec{u} \bmod q$. Output the secret key $\sk_{\vec{v}} = \vec{r}_{\vec{v}} = (\vec{r}_0,..., \vec{r}_{t'})$.
\end{description}
By the property of algorithm $\sampleright$ stated in Lemma~\ref{lem:samp}, we have that the distribution of secret keys generated using $\si.\keygen$ is statistically close to secret keys generated using algorithm $\sampleleft$ in the scheme, both from distribution $\D_{\Lambda_q^{\vec{u}}(\mat{F}), s}$, where $\mat{F} = [\mat{A} | \mat{B}_{\vec{v}_1} | \cdots | \mat{B}_{\vec{v}_{t'}}]$.

The rest of the proof is similar to the $\log(\secparam)$-length one, thus we omit the details here.

\end{appendix}


\end{document}







